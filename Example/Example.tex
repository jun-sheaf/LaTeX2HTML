\documentclass{amsart}

% Preamble would usually go here

\begin{document}
\begin{preliminary}
    Some custom environment
\end{preliminary}
\begin{exercise}
    \begin{ecomponents}
        \item This is
        \item A custom
        \begin{ecomponents}
            \item enumerate
            \item environment
            \begin{enumerate}
                \item with a
                \item[label]
            \end{enumerate}
        \end{ecomponents}
    \end{ecomponents}
    This exercise environment is also custom.
\end{exercise}
Of course math works:
\begin{equation}\label{eq:1}
    \cdots\to \pi_{n}(\S^{n+k})\to \pi_n(\S^n)\to \pi_{n-1}(\S^k)\to\pi_{n-1}(\S^{n+k})\to\cdots
\end{equation}
The standard centered
$$
    \cdots\to \pi_1(\S^n)\to \pi_{0}(\S^0)\to\pi_{0}(\S^{n})\to\pi_{0}(\S^{n})\to 0
$$
and inline $n=4$ math delimiters work. Including \(test\) and
\[\pi_1(S)\]

Here is a typical theorem environment with a label:
\begin{proposition}[Hopf Construction]
Suppose $\mu:X\times X\to X$ is a map such that $\mu(x,\bullet):X\to X$ and $\mu(\bullet,x):X\to X$ are homeomorphisms for each $x\in X$. Then the map
$$
    h_{\mu}:X*X\to \Sigma X\mbox{ defined by }(x,y,t)\mapsto (\mu(x,y),t)
$$
is a fiber bundle with fiber $X$.
\end{proposition}
\begin{proof}
    See \cite[Proposition VII.8.8]{Bredon1993}; we make some remarks. The open cones $C_0$ and $C_1$ defined are the preimages of $(0,1]$ and $[0,1)$ of the map $\Sigma X\to I$ given by projecting the smashing parameter. The map $L_x^{-1}(y)$ can be denoted more simply as $x^{-1}y$ and similarly $yx^{-1}$ for $R_x^{-1}(y)$. The trivialization shown is for $C_1$. $C_2$ admits a similar trivialization with $yx^{-1}$.
\end{proof}

Here is one without a label:\footnote{this is a test footnote}
\begin{proposition}
Suppose $\mu:X\times X\to X$ is a map such that $\mu(x,\bullet):X\to X$ and $\mu(\bullet,x):X\to X$ are homeomorphisms for each $x\in X$. Then the map
$$
    h_{\mu}:X*X\to \Sigma X\mbox{ defined by }(x,y,t)\mapsto (\mu(x,y),t)
$$
is a fiber bundle with fiber $X$.
\end{proposition}
\begin{proof}
    See \cite[Proposition VII.8.8]{Bredon1993}; we make some remarks. The open cones $C_0$ and $C_1$ defined are the preimages of $(0,1]$ and $[0,1)$ of the map $\Sigma X\to I$ given by projecting the smashing parameter. The map $L_x^{-1}(y)$ can be denoted more simply as $x^{-1}y$ and similarly $yx^{-1}$ for $R_x^{-1}(y)$. The trivialization shown is for $C_1$. $C_2$ admits a similar trivialization with $yx^{-1}$.
\end{proof}

Got a figure? Well no problem! This handles figures (rigidly of course) [you can't change your figure position in your original TeX file without modifying the HTML output, but I will add this]
\begin{figure}
    \centering
    \includegraphics[width=4in]{HomotopyOfHomotopy.jpg}
    \caption{If $X=I$, then we have a box $I\times I\times I$ with $A$ being some interval. The flayling nature of $A$ represents the fact the homotopy does not necessarily fix $A$. Going from top to bottom is the homotopy of a homotopy with $\tilde h(x,t)$ along the frame.}\label{fig:2}
\end{figure}

References (e.g. Figure \ref{fig:2}) of course work.
\end{document} 