\documentclass{amsart}

%%%%%%% Packages %%%%%%%

% Latex3 Fix
\RequirePackage{xpatch}

% AMS Packages
\RequirePackage{amsmath,amsfonts,amssymb,amsthm}

% Miscellaneous
\RequirePackage{enumitem, graphicx, geometry}

\RequirePackage[hidelinks]{hyperref}
%%%%%%%%%%%%%%%%%%%%%%%%%

% Commands
\def\S{\mathbb{S}}

% Theorems
\theoremstyle{plain}

\newtheorem{theorem}{Theorem}[section]
\newtheorem{proposition}[theorem]{Proposition}

% Exercise Environment
\newcounter{excount}
\stepcounter{excount}

\NewDocumentCommand
{\exercise}
{ o d() }
{\par\noindent\phantomsection{\bf Exercise %
    \IfNoValueTF{#1}%
    {\arabic{excount}\addcontentsline{toc}{chapter}{Exercise \arabic{excount}}\stepcounter{excount}}%
    {#1%
      \IfNoValueTF{#2}%
      {\addcontentsline{toc}{chapter}{Exercise #1}}
      { (#2)\addcontentsline{toc}{chapter}{Exercise #1 (#2)}}%
    }}.}%

\def\preliminary{\par\noindent\phantomsection{\bf Preliminaries. }\addcontentsline{toc}{chapter}{Preliminaries}}

\def\endexercise{\par\medskip}
\def\endpreliminary{\par\medskip}

% Patching document to include excess macros
\xapptocmd{\document}{\maketitle\tableofcontents}{}{}
\xpretocmd{\enddocument}{\bibliographystyle{plain}\addtocontents{toc}{\vspace{\normalbaselineskip}}\bibliography{References}}{}{}

\title{Example TeX File}
\begin{document}
\begin{preliminary}
    Custom environment.
\end{preliminary}
\begin{exercise}
    This exercise environment is also custom.
    \begin{enumerate}
        \item This is a enumerate environment.
        \item
        \begin{enumerate}[font=\it]
            \item Nested, and changed using the options provided by enumitem.
            \item[Some label] Item with a label.
        \end{enumerate}
    \end{enumerate}
\end{exercise}
Of course math works:
\begin{equation}\label{eq:1}
    \cdots\to \pi_{n}(\S^{n+k})\to \pi_n(\S^n)\to \pi_{n-1}(\S^k)\to\pi_{n-1}(\S^{n+k})\to\cdots
\end{equation}
The standard centered
$$
    \cdots\to \pi_1(\S^n)\to \pi_{0}(\S^0)\to\pi_{0}(\S^{n})\to\pi_{0}(\S^{n})\to 0
$$
and inline $n=4$ math delimiters work. Including \(test\) and
\[\pi_1(S)\]

Here is a typical theorem environment with a label:
\begin{proposition}[Hopf Construction]
Suppose $\mu:X\times X\to X$ is a map such that $\mu(x,\bullet):X\to X$ and $\mu(\bullet,x):X\to X$ are homeomorphisms for each $x\in X$. Then the map
$$
    h_{\mu}:X*X\to \Sigma X\mbox{ defined by }(x,y,t)\mapsto (\mu(x,y),t)
$$
is a fiber bundle with fiber $X$.
\end{proposition}
\begin{proof}
    Omitted.
\end{proof}

Here is one without a label (and a reference label):\footnote{this is a test footnote}
\begin{proposition}\label{prop:1}
Suppose $\mu:X\times X\to X$ is a map such that $\mu(x,\bullet):X\to X$ and $\mu(\bullet,x):X\to X$ are homeomorphisms for each $x\in X$. Then the map
$$
    h_{\mu}:X*X\to \Sigma X\mbox{ defined by }(x,y,t)\mapsto (\mu(x,y),t)
$$
is a fiber bundle with fiber $X$.
\end{proposition}
\begin{proof}
    Omitted
\end{proof}

We can handle labeled citations such as \cite[Proposition VII.8.8]{Bredon1993} and unlabeled such as \cite{Bredon1993}

The following is a figure environment defined like the other environments in this document, but defined drastically differently in css.
\begin{figure}
    \includegraphics[width=4in]{image.jpg}
    \caption{If $X=I$, then we have a box $I\times I\times I$ with $A$ being some interval. The flailing nature of $A$ represents the fact the homotopy does not necessarily fix $A$. Going from top to bottom is the homotopy of a homotopy with $\tilde h(x,t)$ along the frame.}\label{fig:2}
\end{figure}
Note the captions and label are parsed correctly. The graphic is probably empty unless you include a link.

This line is a test that the program can handle sequences and nestings of macros: \textit{face\textbf{map}} of a \textit{test\textit{test}}\footnote{BS} \textbf{space}\footnote{More BS}. \cite[Proposition VII.8.8]{Bredon1993} \cite[Proposition VII.8.8]{Bredon1993}

We can reference our Figure \ref{fig:2} and Proposition \ref{prop:1}.
\end{document} 